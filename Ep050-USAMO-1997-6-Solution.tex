\documentclass[11pt]{scrartcl}
\usepackage{evan}

\begin{document}
\title{USAMO 1997/6}
\subtitle{Evan Chen}
\author{Twitch Solves ISL}
\date{Episode 50}
\maketitle

\section*{Problem}
Suppose the sequence of nonnegative integers
$a_1, a_2, \ldots, a_{1997}$ satisfies
\[ a_i + a_j \leq a_{i+j} \leq a_i + a_j + 1  \]
for all $i,j \geq 1$ with $i + j \leq 1997$.
Show that there exists a real number $x$ such that
$a_n = \lfloor nx \rfloor$ for all $1 \leq n \leq 1997$.

\section*{Video}
\href{https://www.youtube.com/watch?v=q4n-74-t1xY&list=PLi6h8GM1FA6yHh4gDk_ZYezmncU1EJUmZ}{\texttt{https://youtu.be/q4n-74-t1xY}}

\newpage

\section*{Solution}
We are trying to show there exists an $x \in \RR$
such that
\[ \frac{a_n}{n} \le x < \frac{a_n+1}{n} \qquad \forall n. \]
This means we need to show
\[ \max_i \frac{a_i}{i} < \min_j \frac{a_j+1}{j}. \]
Replace 1997 by $N$.
We will prove this by induction,
but we will need some extra hypotheses on the indices $i,j$
which are used above.

\begin{claim*}
	Suppose that
	\begin{itemize}
		\ii Integers $a_1$, $a_2$, \dots, $a_N$ satisfy the given conditions.
		\ii Let $i = \argmax_n \frac{a_n}{n}$;
		if there are ties, pick the smallest $i$.
		\ii Let $j = \argmin_n \frac{a_n+1}{n}$;
		if there are ties, pick the smallest $j$.
	\end{itemize}
	Then \[ \frac{a_i}{i} < \frac{a_j+1}{j}. \]
	Moreover, these two fractions are in lowest terms,
	and are adjacent in the Farey sequence of order $\max(i,j)$.
\end{claim*}
\begin{proof}
	By induction on $N \ge 1$ with the base case clear.
	So suppose we have the induction hypothesis
	with numbers $a_1$, \dots, $a_{N-1}$,
	with $i$ and $j$ as promised.

	Now, consider the new number $a_N$.
	We have two cases:
	\begin{itemize}
		\ii Suppose $i+j > N$.
		Then, no fraction with denominator $N$
		can lie strictly inside the interval;
		so we may write for some integer $b$
		\[ \frac bN \le \frac{a_i}{i}
			< \frac{a_j+1}{j} \le \frac{b+1}{N}. \]
		For purely algebraic reasons we have
		\[ \frac{b-a_i}{N-i} \le \frac bN \le \frac{a_i}{i}
			< \frac{a_j+1}{j} \le \frac{b+1}{N}
			\le \frac{b-a_j}{N-j}. \]
		Now,
		\begin{align*}
			a_N &\ge a_i + a_{N-i}
				\ge a_i + (N-i) \cdot \frac{a_i}{i} \\
			&\ge a_i + (b-a_i) = b \\
			a_N &\le a_j + a_{N-j} + 1 
				\le (a_j+1) + (N-j) \cdot \frac{a_j+1}{j} \\
			&= (a_j+1) + (b-a_j) = b+1.
		\end{align*}
		Thus $a_N \in \{b,b+1\}$.
		This proves that $\frac{a_N}{N} \le \frac{a_i}{i}$
		while $\frac{a_N+1}{N} \ge \frac{a_j+1}{j}$.
		Moreover, the pair $(i,j)$ does not change,
		so all inductive hypotheses carry over.

		\ii On the other hand, suppose $i+j = N$.
		Then we have
		\[ \frac{a_i}{i} < \frac{a_i + a_j + 1}{N} < \frac{a_j+1}{j}.  \]
		Now, we know $a_N$ could be either $a_i + a_j$ or $a_i + a_j + 1$.
		If it's the former, then $(i,j)$ becomes $(i,N)$.
		If it's the latter, then $(i,j)$ becomes $(N,j)$.
		The properties of Farey sequences ensure that
		the $\frac{a_i + a_j + 1}{N}$ is reduced, either way.
	\end{itemize}
\end{proof}

\end{document}
